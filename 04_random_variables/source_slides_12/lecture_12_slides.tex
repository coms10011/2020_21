\documentclass{beamer}
\usepackage[latin1]{inputenc}
%\usetheme{Montpellier}
%\usetheme{Boadilla}
%\usecolortheme[RGB={204,51,255}]{structure}
%\usecolortheme[named=purple]{structure}
\usecolortheme[RGB={128,62,62}]{structure}
%\definecolor{dark}{rgb}{0.3,0.15,0.3}
%\definecolor{light}{rgb}{0.8,0.6,0.8}
%\definecolor{reddish}{rgb}{.5,0.15,0.15}
\definecolor{dark}{rgb}{0.5,0.3,0.4}
%\definecolor{light}{rgb}{0.8,0.6,0.8}
\definecolor{reddish}{rgb}{.7,0.25,0.25}
\definecolor{greenish}{rgb}{.25,0.7,0.25}
\definecolor{blueish}{rgb}{.25,0.25,0.7}
\definecolor{purple}{rgb}{.5,0.0,0.5}
\usepackage{graphicx}
\usepackage{pstricks}

\usepackage{amssymb}

\usepackage{amsmath}
\setbeamertemplate{navigation symbols}{}

\newcommand{\crish}{\color{reddish}}
\newcommand{\cbla}{\color{black}}
\newcommand{\cred}{\color{red}}
\newcommand{\cblu}{\color{blue}}
\newcommand{\cgre}{\color{green}}

\newcommand{\sm}{\color{reddish}$}
\newcommand{\fm}{$\color{black}}
\usepackage{tikz}
\usetikzlibrary{arrows,decorations.markings,positioning}
\usetikzlibrary{calc,fit,shapes, backgrounds} 

\usepackage{epstopdf}
\title{Lecture 12: Expected values}
\author{COMS10014 Mathematics for Computer Science A}
\institute{\texttt{cs-uob.github.io/COMS10014/ and github.com/coms10011/2020\_21}}
\date{November 2020}
\begin{document}

\maketitle

\begin{frame}{Expected values}

If \crish$g(x)$\cbla{} 
is a function we define the \textbf{expected value} of \crish$g(X)$\cbla{}  as
\crish$$
\langle g(X)\rangle = \sum_x p(x)g(x)
$$\cbla{}

\end{frame}


\begin{frame}{Expected values}
If \crish$g(x)$\cbla{} 
is a function we define the \textbf{expected value} of \crish$g(X)$\cbla{}  as
\crish$$
\langle g(X)\rangle = \sum_x p(x)g(x)
$$\cbla{}
or, in another common notation
\crish$$
E[g(X)]=\langle g(X)\rangle
$$\cbla{}
\end{frame}


\begin{frame}{Expected values}
If \crish$g(x)=x$\cbla{} we get the \textbf{expected value} of \crish$X$\cbla{} which is often just called the \textbf{expected value}:
\crish$$
\langle X\rangle = \sum_x xp(x)
$$\cbla{} If \crish$p(x)$\cbla{} is representing the frequencies, then
this is the \textbf{mean}, often called \crish$\mu$\cbla{}.
\end{frame}

\begin{frame}{More names for the same thing}
  
The expect value of \crish$X$\cbla{}  is also referred to as the \textbf{first
  moment}; the \lq{}first\rq{} bit is because it is the expectation
value for the first power of \crish$X$\cbla{} .

\end{frame}

\begin{frame}{Example}
  \crish$X$\cbla{}  is the number of heads if a coin is flipped three times.
  \color{purple}
  \begin{center}
\begin{tabular}{c|cccc}
&0&1&2&3\\
  \hline
  \rule{0pt}{3ex}
$p_X(x)$&$\frac{1}{8}$&$\frac{3}{8}$&$\frac{3}{8}$&$\frac{1}{8}$
\end{tabular}
\end{center}
\cbla{}
\crish$$
\langle X\rangle =0\times\frac{1}{8}+1\times\frac{3}{8}+2\times\frac{3}{8}+3\times \frac{1}{8}=\frac{3+6+3}{8}=\frac{3}{2}
$$\cbla{}
\end{frame}
  
\begin{frame}{Sample mean}
  If we sample multiple times for the sample space and get
  \crish$$\{x_1,x_2,\ldots,x_n\}$$\cbla{}
  as values, when the probabilities are given by \crish$p_X(x)$\cbla{} then the sample mean approaches the expected value:
  \crish$$
  \frac{1}{n}\sum_i x_i \rightarrow \langle X\rangle
  $$\cbla{}
  as \crish$n$\cbla{} goes to infinity.
\end{frame}


\begin{frame}{Sample values}
  If we sample multiple times for the sample space and get
  \crish$$\{x_1,x_2,\ldots,x_n\}$$\cbla{}
  as values, when the probabilities are given by \crish$p_X(x)$\cbla{} then the sample mean approaches the expected value:
  \crish$$
  \frac{1}{n}\sum_i g(x_i) \rightarrow \langle g(X)\rangle
  $$\cbla{}
  as \crish$n$\cbla{} goes to infinity.
\end{frame}



\begin{frame}{Variance}
  The \textbf{variance} is
\crish$$
V(X)=\langle (X-\mu)^2\rangle
$$\cbla{}
When \crish$p(x)$\cbla{}  represents frequencies this is the square of the
\textbf{standard deviation}:
\crish$$V(X)=\sigma^2$$\cbla{}
\end{frame}

\begin{frame}{Variance measures spread}
\color{purple}
  \begin{center}
\begin{tabular}{c|cccc}
&0&1&2&3\\
\hline
$p_X$&1/8&3/8&3/8&1/8
\end{tabular}
\end{center}
  \cbla{}
  has expected value is \crish$1.5$\cbla{} and
  \crish$$
  V(X)=0.75
  $$\cbla{}
\end{frame}


\begin{frame}{Variance measures spread}
\color{purple}
  \begin{center}
\begin{tabular}{c|cccc}
&0&1&2&3\\
  \hline
  $p_Y$&1/16&7/16&7/16&1/16
\end{tabular}
\end{center}
  \cbla{}
  has expected value is \crish$1.5$\cbla{} and
  \crish$$
  V(Y)=0.5
  $$\cbla{}
\end{frame}

\begin{frame}{More names}

\crish$\langle X^2\rangle$\cbla{}  is called the \textbf{second
  moment}, the variance is called the \textbf{second central moment};
the \lq{}central\rq{} indicates that it is the second moment you get
if you take away the mean first.

\end{frame}

\begin{frame}{Other moments}

There are other moments use to describe distributions such as \textbf{skewness} based on the third central moment:
\crish$$
s=\frac{1}{\sigma^3}\langle(X-\mu)^3\rangle
$$\cbla{}
and the \textbf{kurtosis} based on the fourth
\crish$$
\kappa=\frac{1}{\sigma^4}\langle(X-\mu)^4\rangle
$$\cbla{}
\end{frame}


\begin{frame}{Nice properties}
  Scalar multiplication
\crish$$
    \langle c g(X)\rangle =\sum_x cg(x)p(x)=c\sum_x g(x)p(x)=c\langle g(X)\rangle
$$\cbla{}
Also, trivially
\crish$$
\langle 1\rangle=\sum_x p(x)=1
$$\cbla{}
Additive
\crish
\begin{eqnarray*}
\langle g_1(X)+g_2(X)\rangle &=&\sum_x [g_1(x)+g_2(x)]p(x)\cr
&&=\sum_x g_1(x)p(x)+\sum_x g_2(x)p(x)\cr
&&=\langle g_1(X)\rangle+\langle g_2(X)\rangle
\end{eqnarray*}
\cbla{} 
\end{frame}

\begin{frame}{Another formula for variance}
\crish$$
V(X)=\langle (X-\mu)^2\rangle=\langle (X^2-2\mu X+\mu^2)\rangle
$$\cbla{}
Now, using the additive property
\crish$$
V(X)=\langle X^2 \rangle -2\mu\langle X\rangle +\langle \mu^2 \rangle
$$\cbla{}
Finally, noting \crish$\mu=\langle X\rangle$\cbla{},
\crish$$
V(X)=\langle X^2 \rangle - \mu^2 
$$\cbla{}

\end{frame}




\end{document}

