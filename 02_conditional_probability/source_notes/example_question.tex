
\ifind
\section*{Example question}
\else
\subsection*{2 conditional probability - example question}
\fi

This is known as the `second sibling' problem and like the Monty Hall problem it was popularized by Marilyn Vos Savant:
\begin{quote}
 A shopkeeper says she has two new baby beagles to show you, but
    she doesn't know whether they're male, female, or a pair. You tell
    her that you want only a male, and she telephones the fellow who's
    giving them a bath. "Is at least one a male?" she asks him. "Yes!"
    she informs you with a smile. What is the probability that the
    other one is a male?
\end{quote}

\noindent \textbf{solution} So we are assuming that males and females are equally likely and that the sex of the two pups is independent; the set of outcomes, using the obvious notation, is $\mathcal{X}=\{(m,m),(m,f),(f,m),(f,f)\}$ with each event having probability $1/4$. The event that at least one is male is $\mathcal{M}=\{(m,m),(m,f),(f,m)\}$. This means that $P(\mathcal{M})=3/4$. Lets call the event that the second dog is male $\mathcal{S}$ so $\mathcal{M}\cap\mathcal{S}=\{(m,m)\}$ and
\begin{equation}
P(\mathcal{M}\cap\mathcal{S})=\frac{1}{4}
\end{equation}
and hence
\begin{equation}
P(\mathcal{S}|\mathcal{M})=\frac{1}{3}
\end{equation}
and so the probability the second dog is male is one in three.

