%ws5.tex
%notes for the course Mathematics for Computer Science A COMS10014
%taught at the University of Bristol
%2020_21 Conor Houghton conor.houghton@bristol.ac.uk

%To the extent possible under law, the author has dedicated all copyright 
%and related and neighboring rights to these notes to the public domain 
%worldwide. These notes are distributed without any warranty. 


\documentclass[11pt,a4paper]{scrartcl}
\typearea{12}
\usepackage{graphicx}
%\usepackage{pstricks}
\usepackage{listings}
\usepackage{color}
\usepackage{tikz}
\usetikzlibrary{decorations.markings}
\lstset{language=C}
\usepackage{fancyhdr}
\pagestyle{fancy}
\lhead{\texttt{cs-uob.github.io/COMS10014/ and github.com/coms10011/2020\_21}}
\lfoot{COMS10014 - P\&C ws5 - Conor}
\begin{document}

\section*{Probability and Combinatorics Worksheet 5 - outline solutions}


\subsection*{Questions}

These are the questions you should make sure you work on in the workshop.

\begin{enumerate}

\item A typist makes on average two mistakes per page. What is the probability of a particular page having no errors on it?\\
  \\
  \\
  \textbf{Solutions}: So $\lambda=2$ and hence
  \begin{equation}
    p(0)=e^{-2}=0.135
  \end{equation}  

\item Components are packed in boxes of 20. The probability of a
  component being defective is 0.1. What is the probability of a box
  containing 2 defective components?\\ \\ \\ \textbf{Solutions}: This
  is a trick question in the sense that it is binomial distribution
  question not a Poisson distribution question; it is about discrete
  objects, not discrete events in continuous time. So $n=20$ and $p=0.1$ so
  \begin{equation}
    p(2)=\left(\begin{array}{c}20\\2\end{array}\right)0.1^20.9^{18}=0.29
  \end{equation}
 
\item A fisher catches on average one fish every 25 minutes. What is
  the probability that they catch two fish in an hour?
  \\ \\ \\ \textbf{Solution}: If he catches a fish every 25
  minutes then his rate for an hour is 60/25=2.4 so
\begin{equation}
p(0)=\frac{2.4^2}{2}e^{-2.4}\approx 0.26
\end{equation}

 \item A random variable $X$ gives the square of the face value of a six-sided dice. What are the mean and variance of $X$.
   \\ \\ \\ \textbf{Solution}:
   So for the mean it is
   \begin{equation}
     \mu=\frac{1+4+9+16+25+36}{6}=15.17
   \end{equation}
   and the second moment is
   \begin{equation}
     \langle X^2\rangle =\frac{1+16+81+256+625+1296}{6}=379.17
   \end{equation}
   so
   \begin{equation}
     \sigma^2=\langle X^2\rangle-\mu^2=149.1
   \end{equation}
   
\end{enumerate}

\subsection*{Extra questions}
Do these in the workshop if you have time.


\begin{enumerate}

\item For a Poisson process let $T$ be the interval for which the
process has, on average, one event, so, for this interval $\lambda=1$. What is the probability that there are no events for this interval?\\ \\ \\  
\textbf{Solution}
\begin{equation}
p(0)=\frac{1}{e}=0.37
\end{equation}

\item Starting with the expression for the mean
  \begin{equation}
    \lambda =\sum_{r=0}^\infty \frac{\lambda^r}{r!}re^{-lambda}
  \end{equation}
  calculate the variance of the Poisson distribution.\\ \\ \\  
  \textbf{Solution}: So differentiate both sides with respect to $\lambda$:
  \begin{equation}
    1=\sum_{r=0}^\infty r^2 \frac{\lambda^{r-1}}{r!}e^{-lambda}-\sum_{r=0}^\infty \frac{\lambda^r}{r!}re^{-lambda}
  \end{equation}
  We do the usual trick of multiplying and dividing by $\lambda$ in the first term, and use the fact the second term is just the mean:
  \begin{equation}
    1=\frac{1}{\lambda}\sum_{r=0}^\infty r^2 \frac{\lambda^r}{r!}e^{-lambda}-\lambda
  \end{equation}
  Now, note the first term is the second moment, also multiply across by $\lambda$:
  \begin{equation}
    \lambda=\langle R^2\rangle-\lambda^2
  \end{equation}
  and since $\mu=\lambda$ the righthand side is $\sigma^2$ so
  \begin{equation}
    \sigma^2=\lambda
  \end{equation}
  
  
\item The \textbf{Fano factor} is sometimes used to describe distributions, it is
  \begin{equation}
    F=\frac{\sigma^2}{\mu}
  \end{equation}
What is the Fano factor for the Poisson distribution?
\\ \\ \\  
\textbf{Solution}: well the mean and variance are both $\lambda$ so this is one.

\end{enumerate}


\end{document}

