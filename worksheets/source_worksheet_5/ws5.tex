%ws5.tex
%notes for the course Mathematics for Computer Science A COMS10014
%taught at the University of Bristol
%2020_21 Conor Houghton conor.houghton@bristol.ac.uk

%To the extent possible under law, the author has dedicated all copyright 
%and related and neighboring rights to these notes to the public domain 
%worldwide. These notes are distributed without any warranty. 


\documentclass[11pt,a4paper]{scrartcl}
\typearea{12}
\usepackage{graphicx}
%\usepackage{pstricks}
\usepackage{listings}
\usepackage{color}
\usepackage{tikz}
\usetikzlibrary{decorations.markings}
\lstset{language=C}
\usepackage{fancyhdr}
\pagestyle{fancy}
\lhead{\texttt{cs-uob.github.io/COMS10014/ and github.com/coms10011/2020\_21}}
\lfoot{COMS10014 - P\&C ws5 - Conor}
\begin{document}

\section*{Probability and Combinatorics Worksheet 5}

\subsection*{Useful facts}

\begin{itemize}

\item \textbf{Expected value}. For a discrete random variable with probability $p(x)$ this is
\begin{equation}
\langle g(X) \rangle = \sum_x p(x)g(x)
\end{equation}
For a continuous random variable with density $f(x)$ this is
\begin{equation}
\langle g(X)\rangle = \int_{-\infty}^\infty{f(x)g(x)}dx
\end{equation}


\item \textbf{Mean and variance}. The mean is $\mu=\langle X\rangle$ and the variance is $\sigma^2=\langle(X-\mu)^2\rangle=\langle X^2\rangle - \mu^2$.


\item \textbf{Poisson distribution}. This has
\begin{equation}
p(r)=\frac{\lambda^r}{r!}e^{-\lambda}
\end{equation}
where $\mu=\lambda$ and $\sigma^2=\lambda$.


\item \textbf{The limit of infinite compounding}
  \begin{equation}
    \left(1-\frac{x}{n}\right)^n\rightarrow e^{-x}
  \end{equation}
  as $n\rightarrow\infty$.

  
\end{itemize}



\subsection*{Questions}

These are the questions you should make sure you work on in the workshop.

\begin{enumerate}

  

\item A typist makes on average two mistakes per page. What is the probability of a particular page having no errors on it?

\item Components are packed in boxes of 20. The probability of a component being defective is 0.1. What is the probability of a box containing 2 defective components?
  
\item A fisher catches on average one fish every 25 minutes. What is
  the probability that they catch two fish in an hour?

 \item A random variable $X$ gives the square of the face value of a six-sided dice. What are the mean and variance of $X$.

\end{enumerate}

\subsection*{Extra questions}
Do these in the workshop if you have time.


\begin{enumerate}

\item For a Poisson process let $T$ be the interval for which the
process has, on average, one event, so, for this interval $\lambda=1$. What is the probability that there are no events for this interval?  

\item Starting with the expression for the mean
  \begin{equation}
    \lambda =\sum_{r=0}^\infty \frac{\lambda^r}{r!}re^{-lambda}
  \end{equation}
  calculate the variance of the Poisson distribution.
  
\item The \textbf{Fano factor} is sometimes used to describe distributions, it is
  \begin{equation}
    F=\frac{\sigma^2}{\mu}
  \end{equation}
What is the Fano factor for the Poisson distribution?

\end{enumerate}

\subsection*{Another question}

There is no need to do this question and it won't be considered in the
workshop. A Poisson process represents true randomness in the sense
that each event is unrelated to all the others. Consider a discretized
two-dimensional Poisson process where the squares on a grid are black
with a probability $p$ and white with probability $1-p$; if you
generate a grid like this does the distribution of black squares look
random?




\end{document}

