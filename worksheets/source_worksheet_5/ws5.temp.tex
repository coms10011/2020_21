%ws3.tex
%notes for the course Mathematics for Computer Science A COMS10014
%taught at the University of Bristol
%2020_21 Conor Houghton conor.houghton@bristol.ac.uk

%To the extent possible under law, the author has dedicated all copyright 
%and related and neighboring rights to these notes to the public domain 
%worldwide. These notes are distributed without any warranty. 


\documentclass[11pt,a4paper]{scrartcl}
\typearea{12}
\usepackage{graphicx}
%\usepackage{pstricks}
\usepackage{listings}
\usepackage{color}
\usepackage{tikz}
\usetikzlibrary{decorations.markings}
\lstset{language=C}
\usepackage{fancyhdr}
\pagestyle{fancy}
\lhead{\texttt{cs-uob.github.io/COMS10014/ and github.com/coms10011/2020\_21}}
\lfoot{COMS10014 - P\&C ws3 - Conor}
\begin{document}

\section*{Worksheet Sheet 2 - outline solutions}

\subsection*{Questions}

\begin{enumerate}


\item The illusionist Derren Brown famously flipped a coin on camera
  so that it landed heads ten times in a row; he claimed that this was
  because of his mind powers, in fact it was because of his patience,
  he simply kept trying the trick again and again until it
  worked. It took him nine hours. What is the probability of a coin landing heads ten times in
  a row? If you flip a coin ten times what is the probability of
  getting five heads and five tails?\\ \\ \\ \textbf{Solution}: The chance of 10 heads is
\begin{equation}
p(10)=0.5^{10}=0.0009765625
\end{equation}
whereas the chance of five heads is
\begin{equation}
p(5)=\left(\begin{array}{c}10\\5\end{array}\right)0.5^{10}=0.24609375
\end{equation}


\item You are a con person with a crooked coin, it has a 0.25 chance
  of a head and a 0.75 chance of a harp. You make a bet with someone; if
  the coin shows a head you pay them two pounds, if it shows a harp
  they give you one. What is the mean return from a bet; what is the variance?
\\ \\ \\
\textbf{Solution}: So 
\begin{equation}
\mu =-2\times 0.25 +1\times 0.75=0.25
\end{equation}
and if $X$ is your winnings
\begin{equation}
  \langle X^2\rangle=4\times 0.25 +1\times 0.75=1.75
\end{equation}
so $\sigma^2=\langle X^2\rangle-\mu^2=1.6875$
  
\item For the binomial distribution on $n$ trials, prove
  $\sigma^2=npq$; this can be done by differenciating $Z$ twice, an
  easier approach is to differenciate the expression for $\mu=np$:
  \begin{equation}
    np=\sum_{r=0}^n \left(\begin{array}{c}n\\r\end{array}\right)rp^rq^{n-r}
  \end{equation}
\\ \\ \\ 
\textbf{Solution}: So
\begin{equation}
    np=\sum \sum_{r=0}^n \left(\begin{array}{c}n\\r\end{array}\right)rp^rq^{n-r}
\end{equation}
and differentiating both sides and then doing the multiplying and dividing by $p$ or $q$ trick gives
\begin{equation}
n=\frac{1}{p}\sum_{r=0}^n r^2p_R(r)-\frac{1}{q}\sum_{r=0}^n r(n-r)p_R(r)
\end{equation}
or
\begin{equation}
n=\left(\frac{1}{p}+\frac{1}{q}\right)\langle R^2\rangle-\frac{n}{q}\langle R\rangle
\end{equation}
and after a bit of algebra and using $\langle R\rangle =np$ we get
\begin{equation}
  npq=\langle R^2\rangle -n^2p^2
\end{equation}
and we note that the right hand side is $\langle R^2\rangle-\langle R\rangle^2$ which is $\sigma^2$.

\item Like the binomial distribution the geometric probability
  distribution is related to a series of independent trials where each
  trial has probability $p$ of success and $q=1-p$ of failure. The
  geometric probability $p(r)$ is the probability that the $r$th trial
  is the first success. It is
\begin{equation}
p(r)=q^{r-1}p
\end{equation}
It can be shown that 
\begin{equation}
\sum_{r=1}^\infty p(r)=1
\end{equation}
as it must be. You can assume that here. What is the mean of the geometric probability? As a hint, this is done much as for the binomial expansion.
\\ \\ \\ 
\textbf{Solution}: So we have
\begin{equation}
1=\sum_{r=1}^\infty q^{r-1}p
\end{equation}
If we differenciate both sides by $p$ we get
\begin{equation}
0=\sum_{r=1}^\infty q^{r-1}-\sum_{r=1}^\infty (r-1)q^{r-2}p
\end{equation}
or
\begin{equation}
0=\frac{1}{p}\sum_{r=1}^\infty q^{r-1}p-\sum_{r=1}^\infty (r-1)q^{r-2}p
\end{equation}
In the second term set $s=r-1$ to get 
\begin{equation}
\sum_{r=1}^\infty (r-1)q^{r-2}p=\sum_{s=0}^\infty sq^{s-1}p=\sum_{s=1}^\infty sq^{s-1}p
\end{equation}
where we are able to change drop the $s=0$ term in the sum because it gives zero. Now back to the original equation:
\begin{equation}
0=\frac{1}{p}\sum_{r=1}^\infty q^{r-1}p-\sum_{s=1}^\infty sq^{s-1}p
\end{equation}
Now note that the first sum gives one and the second gives $\mu$ so
\begin{equation}
\mu=\frac{1}{p}
\end{equation}

We weren't asked to prove
\begin{equation}
\sum_{r=1}^\infty p(r)=1
\end{equation}
but in case you are interested it is discussed here
\begin{equation}
\sum_r q^{r-1}p=1
\end{equation}
For this we use the expansion
\begin{equation}
\frac{1}{1-q}=1+q+q^2+\ldots
\end{equation}
you can think of this as coming from the binomial expansion of $(1-q)^{-1}$ using the generalized binomial formula discovered by Newton
\begin{equation}
(1+x)^n=\sum_r \frac{n(n-1)\ldots (n-r+1)}{r!}x^r
\end{equation}
with $n=-1$; it can also be derived as the Taylor expansion. Either way we have
\begin{equation}
\frac{1}{1-q}=\sum_{r=0}^\infty q^r=\sum_{r=1}^\infty q^{r-1}
\end{equation}
and since $1-q=p$ this gives the result.

\end{enumerate}

\subsection*{Extra questions}

\begin{enumerate}

\item Oranmore, the village I grew up in had more people with the
  surname Furey than any other village or town in the world. Since I
  left the village has expanded ten-fold and has gone from being a
  small village to a commuter town for Galway. However, when I was young
  one in ten people in the village had the surname Furey. Imagine
  there are 35 children in a class at school, what is the probability
  that five of them were Fureys?

  \textbf{Solution}: So this is just
  \begin{equation}
    p(r=5)=\left(\begin{array}{c}35\\5\end{array}\right) \left(\frac{1}{10}\right)^{5}\left(\frac{9}{10}\right)^{30}
\end{equation}
which is about $0.14$.
  
\item The \textbf{Fano factor} is sometimes used to describe distributions, it is
  \begin{equation}
    F=\frac{\sigma^2}{\mu}
  \end{equation}
What is the Fano factor for the Poisson distribution?\\
\\ \\
\textbf{Solution}: Well the mean is $np$ and the variance is $npq$ so $F=q$.

\item What is the variance of the geometric distribution?\\ \\ \\
  \textbf{Solution}: Well we have
  \begin{equation}
    \frac{1}{p}=\sum_{r=0}^\infty rq^{r-1}p
  \end{equation}
  so differentiating both sides gives
  \begin{equation}
    -\frac{1}{p^2}=-\sum_{r=0}^\infty r(r-1)q^{r-2}p+\sum_{r=0}^\infty rq^{r-1}
  \end{equation}
  and so with a bit of algebra we get
  \begin{equation}
    -\frac{1}{p^2}=-\frac{1}{q}\langle R^2\rangle+\left(\frac{1}{q}+\frac{1}{p}\right)\langle R\rangle
  \end{equation}
  and substituting for $\langle R\rangle=1/p$ and doing still more algebra we get
  \begin{equation}
    \sigma^2=\frac{q}{p^2}
  \end{equation}

\item The negative binomial distribution models a binomial process
  where, instead of $n$ trials, the experiment continues until there
  are $r$ failures. The probability of $k$ successes is
  \begin{equation}
    p_K(k)=\left(\begin{array}{c}k+r-1\\k\end{array}\right)(1-p)^rp^k
  \end{equation}
Obviously if
  \begin{equation}
    Z=\sum_{k=0}^\infty p_K(k)
  \end{equation}
  then $Z=1$; this relies on the negative binomial expansion and
  you don't need to do this calculation. By differenciating calculate
  the mean of the negative binomial distribution.\\
\\ \\
\textbf{Solution}: It's just the usual craic, we differentiate $Z$, for the same of space lets write
\begin{equation}
  C_k=\left(\begin{array}{c}k+r-1\\k\end{array}\right)
\end{equation}
and we get
\begin{equation}
  0=\sum_{k=0}^\infty C_k k (1-p)^rp^{k-1}-\sum_{k=0}^\infty C_k r(1-p)^{r-1}p^k
\end{equation}
and so
\begin{equation}
  \frac{1}{p}\sum_{k=0}^\infty C_k k q^rp^k=\frac{1}{q}\sum_{k=0}^\infty C_k rq^rp^k
\end{equation}
or
\begin{equation}
  \langle K\rangle = \frac{pr}{q}
\end{equation}


\end{enumerate}
  

\end{document}

