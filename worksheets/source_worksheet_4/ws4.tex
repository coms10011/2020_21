%ws3.tex
%notes for the course Mathematics for Computer Science A COMS10014
%taught at the University of Bristol
%2020_21 Conor Houghton conor.houghton@bristol.ac.uk

%To the extent possible under law, the author has dedicated all copyright 
%and related and neighboring rights to these notes to the public domain 
%worldwide. These notes are distributed without any warranty. 


\documentclass[11pt,a4paper]{scrartcl}
\typearea{12}
\usepackage{graphicx}
%\usepackage{pstricks}
\usepackage{listings}
\usepackage{color}
\usepackage{tikz}
\usetikzlibrary{decorations.markings}
\lstset{language=C}
\usepackage{fancyhdr}
\pagestyle{fancy}
\lhead{\texttt{cs-uob.github.io/COMS10014/ and github.com/coms10011/2020\_21}}
\lfoot{COMS10014 - P\&C ws3 - Conor}
\begin{document}

\section*{Probability and Combinatorics Worksheet 4}

\subsection*{Useful facts}

\begin{itemize}

\item \textbf{Expected value}. For a discrete random variable with probability $p(x)$ this is
\begin{equation}
\langle g(X) \rangle = \sum_x p(x)g(x)
\end{equation}
For a continuous random variable with density $f(x)$ this is
\begin{equation}
\langle g(X)\rangle = \int_{-\infty}^\infty{f(x)g(x)}dx
\end{equation}


\item \textbf{Mean and variance}. The mean is $\mu=\langle X\rangle$ and the variance is $\sigma^2=\langle(X-\mu)^2\rangle=\langle X^2\rangle - \mu^2$.

\item \textbf{Binomial distibution}. For $n$ independent trials each with $p$ change of success and $q=1-p$ of failure, the probability of $r$ successes is 
\begin{equation}
p(r)=\left(\begin{array}{c}n\\r\end{array}\right)p^rq^{n-r}
\end{equation}
and $\mu=pn$, $\sigma^2=pqn$.

\end{itemize}



\subsection*{Questions}

These are the questions you should make sure you work on in the workshop.

\begin{enumerate}

\item The illusionist Derren Brown famously flipped a coin on camera
  so that it landed heads ten times in a row; he claimed that this was
  because of his mind powers, in fact it was because of his patience,
  he simply kept trying the trick again and again until it
  worked. It took him nine hours. What is the probability of a coin landing heads ten times in
  a row? If you flip a coin ten times what is the probability of
  getting five heads and five tails?

\item You are a con person with a crooked coin, it has a 0.25 chance
  of head and a 0.75 change of a harp. You make a bet with someone; if
  the coin shows a head you pay them two points, if it shows a harp
  they give you one. What is the mean return from a bet; what is the variance.

\item For the binomial distribution on $n$ trials, prove
  $\sigma^2=npq$; this can be done by differenciating $Z$ twice, an
  easier approach is to differenciate the expression for $\mu=np$:
  \begin{equation}
    np= \sum_{r=0}^n \left(\begin{array}{c}n\\r\end{array}\right)rp^rq^{n-r}
  \end{equation}
  
\item Like the binomial distribution the geometric probability
  distribution is related to a series of independent trials where each
  trial has probability $p$ of success and $q=1-p$ of failure. The
  geometric probability $p(r)$ is the probability that the $r$th trial
  is the first success. It is
\begin{equation}
p(r)=q^{r-1}p
\end{equation}
It can be shown that 
\begin{equation}
\sum_{r=1}^\infty p(r)=1
\end{equation}
as it must be. You can assume that here. What is the mean of the geometric probability?



\end{enumerate}

\subsection*{Extra questions}
Do these in the workshop if you have time.

\begin{enumerate}

\item Oranmore, the village I grew up in had more people with the
  surname Furey than any other village or town in the world. Since I
  left the village has expanded ten-fold and has gone from being a
  small village to a commuter town for Galway. However, when I was young
  one in ten people in the village had the surname Furey. Imagine
  there are 35 children in a class at school, what is the probability
  that five of them were Fureys?

\item The \textbf{Fano factor} is sometimes used to describe distributions, it is
  \begin{equation}
    F=\frac{\sigma^2}{\mu}
  \end{equation}
What is the Fano factor for the binomial distribution?

\item What is the variance of the geometric distribution?

\item The negative binomial distribution models a binomial process
  where, instead of $n$ trials, the experiment continues until there
  are $r$ failures. The probability of $k$ successes is
  \begin{equation}
    p_K(k)=\left(\begin{array}{c}k+r-1\\k\end{array}\right)(1-p)^rp^k
  \end{equation}
Obviously if
  \begin{equation}
    Z=\sum_{k=0}^\infty p_K(k)
  \end{equation}
  then $Z=1$; this relies on the negative binomial expansion and
  you don't need to do this calculation. By differenciating calculate
  the mean of the negative binomial distribution.
\end{enumerate}

\end{document}

