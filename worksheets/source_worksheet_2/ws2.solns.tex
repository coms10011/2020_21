%ps1.solns.tex
%notes for the course Probability and Statistics COMS10011 
%taught at the University of Bristol
%2018_19 Conor Houghton conor.houghton@bristol.ac.uk

%To the extent possible under law, the author has dedicated all copyright 
%and related and neighboring rights to these notes to the public domain 
%worldwide. These notes are distributed without any warranty. 

\documentclass[11pt,a4paper]{scrartcl}
\typearea{12}
\usepackage{graphicx}
%\usepackage{pstricks}
\usepackage{listings}
\usepackage{color}
\lstset{language=C}
\usepackage{fancyhdr}
\pagestyle{fancy}
\lhead{\texttt{cs-uob.github.io/COMS10014/ and github.com/coms10011/2020\_21}}
\lfoot{COMS10014 - P\&C ws2 solns - Conor}
\begin{document}

\section*{Worksheet 2 - outline solutions}

\subsection*{Questions}

\begin{enumerate}
\item How many distinct anagrams has the word `BOOKKEEPER'?\\ \\ \\
  \textbf{Solution}: So ten slots to split into a group of three, for the `E's, two groups of two for the 'O's and the 'K's and three groups of one for the 'B', 'P' and 'R':
  \begin{equation}
      n=\left(\begin{array}{c}10\\3,2,2,1,1,1\end{array}\right)-1=\frac{10!}{3!2!2!}-1=151199
  \end{equation}

  
\item Two events $A$ and $B$ have probabilities $P(A)=0.2$, $P(B)=0.3$ and $P(A\cup B)=0.4$. Find
\begin{enumerate}
\item Find $P(A\cap B)$.
\item Find $P(\bar{A}\cap \bar{B})$.
\item Find $P(A|B)$.
\end{enumerate}
\textbf{Solution}: So
\begin{equation}
P(A\cup B)=P(A)+P(B)-P(A\cap B)
\end{equation}
so 
\begin{equation}
P(A\cap B)=0.2+0.3-0.4=0.1
\end{equation}
If $P(A\cap B)=0.1$ then $P(\bar{A}\cap B)=P(B)-P(A\cap B)=0.2$. One the other hand
$P(\bar{A}\cap \bar{B})=1-P(A\cup B)=0.6$. Finally
\begin{equation}
P(A|B)=\frac{P(A\cap B)}{P(B)}=\frac{0.1}{0.3}=\frac{1}{3}
\end{equation}

\item How many five move games of tic-tac-toe are there?\\ \\ \\
  \textbf{Solution}: So to win X must have a line, there are eight lines, so the choice of line gives $8$ and the order X filled it gives $3!$, O has two places picked one by one from six free slots, so that's $6\times 5$ and
  \begin{equation}
    n=8\times 3!\times 30 = 1440
  \end{equation}


\item You roll a dice twice, what is the probability the second roll
  has a lower value than the first? You take the ace, two, three,
  four, five and six of hearts to make a mini-pack of cards. You draw
  two cards, what is the probability the second will be lower than the
  first, with the ace counting as a one?\\ \\ \\ \textbf{Solution}:
  Dice first: imagine a grid of outcomes, so you tick entry $x_{ij}$
  if the first roll is $i$ and the second is $j$; all entries are
  equally likely so we just want to count the number of elements such
  that $i>j$, there are 15 such elements in a 36 element grid so the
  probability is $15/36=5/12$. Now the cards, this doesn't seem so
  easy since the first card is removed but if you think of the same
  grid, this just means that there are no diagonal elements and hence
  we have $15/30=1/2$. If this seems implausible you can calculate it
  the longer, but more intuitive way:
  \begin{equation}
    P=\frac{1}{6}+\frac{1}{6}\frac{4}{5}+\frac{1}{6}\frac{3}{5}+\frac{1}{6}\frac{2}{5}+\frac{1}{6}\frac{1}{5}
  \end{equation}
  where the first entry corresponds to the first pick being a six, the
  second to the first pick being a five and so on. Doing this sum
  gives a half.

\end{enumerate}
  
\subsection*{Extra questions}

\begin{enumerate}

\item How many paths are there from (0,0) to (4,5)? If you pick a random path from (0,0) to (4,5) what is the probability it goes through $(2,2)$?\\ \\ \\
  In all you make $4+5=9$ moves and you need to pick four of them to be across, so that
  \begin{equation}
    n=\left(\begin{array}{c}9\\4\end{array}\right)=126
  \end{equation}
  paths. If the path goes through $(2,2)$ then we first count the paths from $(0,0)$ to $(2,2)$
  \begin{equation}
    n_1=\left(\begin{array}{c}4\\2\end{array}\right)=6
  \end{equation}
  and then from $(2,2)$ to $(4,5)$ which is like going from $(0,0)$ to $(2,3)$
  \begin{equation}
    n_2=\left(\begin{array}{c}5\\2\end{array}\right)=10
  \end{equation}
  so the number of paths going through $(2,3)$ is $n_1+n_2=16$ and the probability of a random path going through $(2,2)$ is
  \begin{equation}
    P=\frac{16}{126}=\frac{8}{63}
  \end{equation}
    
\item How many six move games of tic-tac-toe are
  there?\\ \\ \\ \textbf{Solution}: So as before there are eight ways
  for O to win and $3!$ orders for them to have occupied those three
  slots in the win and $6\times 5\times 4$ plays for X;
  \begin{equation}
    n_1=8\times 3! \time 120 =5760
  \end{equation}
  however, X can't have won at round five for this to count as a six
  move game. We need to count the six move games where X won at round
  five and then O took one more move and produced at round six a game
  we have counted as part of $n_1$ but which shouldn't be included
  when calculating the total number of six round games. In a game like
  this there are six choices for the location of the X line, since
  diagonals don't work, if a diagonal is occupied, no other line is
  possible. Now when X has a row there are two choices for a line of
  O's, the two other rows, the same is true of columns. Finally each of X and O have 3! orders they could've places their marks giving
  \begin{equation}
    n_2=6\times 2\times (3!)^2=432
  \end{equation}
  so
  \begin{equation}
    n=n_1-n_2=5760-432=5328
  \end{equation}


\item If $p$ is a probability mass function and we define a function on events:
  \begin{equation}
    P(E)=\sum_{x\in E}p(x)
  \end{equation}
  show $P$ is a probability.\\ \\ \\ \textbf{Solution}: So the
  properies of a probability are that it is positive, which follows
  from the positivity of the mass function, that $P(X)=1$ which
  follows from 
  \begin{equation}
    \sum_{x\in X}p(x)=1
  \end{equation}
  and $P(A\cup B)=P(A)+P(B)$ if $A\cap B=\emptyset$ follows from
    \begin{equation}
    \sum_{x\in A\cup B}p(x)=    \sum_{x\in A}p(x)+    \sum_{x\in B}p(x)
  \end{equation}
  if $A\cap B=\emptyset$.
  
\end{enumerate}
  


\end{document}

