%3_bayes.tex
%notes for the course Mathematics for Computer Scientists A
%taught at the University of Bristol
%2020_21 Conor Houghton conor.houghton@bristol.ac.uk

%To the extent possible under law, the author has dedicated all copyright 
%and related and neighboring rights to these notes to the public domain 
%worldwide. These notes are distributed without any warranty. 


\documentclass[11pt,a4paper]{scrartcl}
\typearea{12}
\usepackage{graphicx}
%\usepackage{pstricks}


\newif\ifind
\indtrue


\usepackage{listings}
\usepackage{color}
\lstset{language=C}
\usepackage{fancyhdr}
\pagestyle{fancy}
\lfoot{\texttt{coms10011.github.io}}
\rfoot{\texttt{coms10011.github.io}}
\lhead{COMS10014 Bayes Theorem - Conor}
\begin{document}

\section*{3 Bayes' theorem}

\subsection*{Independent events and some laws of probability}

Two events $A$ and $B$ are said to be \textbf{independent} if 
\begin{equation}
P(A\cap B)=P(A)P(B)
\end{equation}
This is equivalent to says that two events are independent if
\begin{equation}
P(A|B)=P(A)
\end{equation}
or 
\begin{equation}
P(B|A)=P(B)
\end{equation}
What it says is that the probability $A$ and $B$ happen together is
just the probability that $A$ happens multiplied by the probability
$B$ happens; so, for example, $A$ happening doesn't change the
probability that $B$ happens.


Here is an example a slightly complicated example. Imagine a very
boring game of like snakes and ladders with no snakes and no
ladders. Every time it is your go you flip a coin, if you get a harp
you go forward one step, if you get the other side you stay put. Hence for example, three
rounds into the game your locations have probabilities
\begin{center}
\begin{tabular}{c|cccc}
&0&1&2&3\\
\hline
$P$&1/8&3/8&3/8&1/8
\end{tabular}
\end{center}
Let $S_3^3$ be the event you are on the third square after three
rounds, so $P(S_3^3)=1/8$ and, so, if $S_3^2$ is the event you are at
the second square after three rounds the $P(S_3^2)=3/8$ because it
corresponds to three possible sequences (H, H, T), (H, T, H) and (T,
H, H), out of eight possible sequences in all. Let $S_2^2$ be the
event that you are at square two after two goes, clearly
$P(S_2^2)=1/4$. Now
\begin{equation}
P(S_3^3|S_2^2)=1/2
\end{equation}
which is different from $P(S_3^3)$ so your position after two moves is
not independent from your position after three. It is similarly true
that your position after three moves is not independent of your
position after one move. 


Before considering Bayes' theorem it is useful to note two laws of probability. The multiplicative law is
\begin{equation}
P(A\cap B)=P(A)P(B|A)
\end{equation}
which follows directly from the definition of conditional probability. The additive law is
\begin{equation}
P(A\cup B)=P(A)+P(B)-P(A\cap B)
\end{equation}
Finally, a corollary of the addative law tells us
\begin{equation}
P(A)=1-P(\bar{A})
\end{equation}
where, recall, $\bar{A}$ is the complement set to $A$, it is all the
outcomes in the sample space that are not in $A$.

\subsection*{Bayes' theorem}

Consider the formula for the conditional probability:
\begin{equation}
P(A|B)=\frac{P(A\cap B)}{P(B)}
\end{equation}
which is equivalent to
\begin{equation}
P(A|B)P(B)=P(A\cap B)
\end{equation}
which tells us that the probability of $A$ and $B$ is the probability
of $B$ multiplied by the probability of $A$ given $B$. This makes lots
of sense, but it is also notable that the left hand side doesn't look
symmetric in $A$ and $B$ while the right hand side clearly
is. Obviously this means we can write
\begin{equation}
P(A|B)P(B)=P(B|A)P(A)
\end{equation}
or
\begin{equation}
P(A|B)=\frac{P(B|A)P(A)}{P(B)}
\end{equation}
This formula is called the Bayes' rule and is surprisingly useful
because there are lots of interesting problems where we are told
$P(B|A)$ but would like to know $P(A|B)$.

Often the example given is related to testing. Lets say 5\% of steaks
sold as beef steak are actually made of horse and imagine we have a
horsiness test which is positive 90\% of the time when tested on horse
and 10\% of the time when tested on beef. If a piece of steak tests
positive for horse, what is the chance it is horse? Let $H$ be the
event of being horse and $Y$ the event of testing positive for
horsiness. Now we know $P(H)=0.05$ and $P(Y|H)=0.9$; what we want is
$P(H|Y)$ and this is what Bayes' rule is useful for:
\begin{equation}
P(H|Y)=\frac{P(Y|H)P(H)}{P(Y)}
\end{equation}

We don't have $P(Y)$ but we can work it out:
\begin{equation}
P(Y)=P(Y|H)P(H)+P(Y|\bar{H})P(\bar{H})
\end{equation}
since $P(Y\cap H)=P(Y|H)P(H)$ and so on. Hence
\begin{equation}
P(Y)=0.9\times 0.05 + 0.1\times 0.95=0.14
\end{equation}
Thus
\begin{equation}
P(H|Y)=\frac{0.9\times 0.05}{0.14}=0.32
\end{equation}
Hence, surprisingly, if a steak tests positive for horsiness it is
still more likely to be beef. Basically, because there are so many
more beef steaks than horse steaks, the relatively small false
positive rate for beef still leads to a reasonably high chance a piece
of steak that tests positive for horse is nonetheless beef.

There is a particular terminology associated with Bayes' rule; it is
sometimes written:
\begin{equation}
\mbox{posterior}=\frac{\mbox{likelihood}\times \mbox{prior}}{\mbox{evidence}}
\end{equation}
The \textsl{posterior} is the probability estimated after the evidence
is gathered, for example, the chance of horsiness after we have found
the test is positive.  The \textsl{likelihood} is how likely the
evidence is given the event, in the example above, it is $P(Y|H)$; the
\textsl{prior} is the probability estimated before the evidence is
gathered, that is $P(H)$, finally \textsl{evidence} measure the
probability of the evidence, $P(Y)$.

\subsection*{Na\"ive Bayes estimator}

Many learning algorithms can be thought of as machines for
estimating probabilities, often in the face of insufficient data to
estimate the probabilities required. A common example used to
illustrate this is a spam filter. Let $W$ represent an ordered list of
words that may be in an email, say:
\begin{equation}
W=(\mbox{enlargement},\mbox{xxx},\mbox{cheapest},\mbox{pharmaceuticals}, \mbox{satisfied},\mbox{leeds})
\end{equation}
It isn't enough to look at these words on their own; an email with the
word \lq{}enlargement\rq{} might be talking about photographs, someone
might actually be from Leeds. For this reason it is more useful to
look at combinations. Say $\textbf{w}$ is a vector of zeros and ones
indicating the presence or absence of different potential spam words
in an email. Thus, an email that includes the words
\lq{}enlargement\lq{}, \lq{}xxx\rq{} and \lq{}leeds\rq{} but not
\lq{}cheapest\rq{}, \lq{}pharmaceuticals\rq{} and \lq{}satisfied\rq{}
would be represented by
\begin{equation}
\textbf{w}=(1,1,0,0,0,1)
\end{equation}
Now let $S$ represent the event of an email being spam. The objective
with a spam filter is to estimate $P(S|\textbf{w})$ for every possible
vector $\textbf{w}$ and then use a cut-off to label any email with a
high probability of being spam as \lq{}spam\rq{}.

Obviously if you have a truly huge amount of data you could estimate this probability by counting:
\begin{equation}
P(S|(1,1,0,0,0,1))=\frac{\#\{\mbox{spam emails with the words enlargement, xxx and leeds}\}}{\#\{\mbox{all emails with the words enlargement, xxx and leeds}\}}
\end{equation}
where by \lq{}spam emails with the words enlargement, xxx and
leeds\rq{} we mean spam emails with those words, but not the three
others, cheapest, pharmaceuticals and satisfied; these correspond to
zeros in the $\textbf{w}$ vector. Now, the problem is there are
$2^6=64$ possible $\textbf{w}$ vectors, and of course in a real
example you'd need many more than six words, thus, for anything but an
infeasibly large data set, the amount of emails with the precise
combination of words represented by a given $\textbf{w}$ will be tiny,
leading to a poor estimate of the probabilities. For example, if there
are 30 words being considered, still nothing like enough to think
about when building a spam filter, then there are just over a billion
possible $\textbf{w}$ vectors; that means for most $\textbf{w}$
vectors the number of spam emails corresponding to $\textbf{w}$ will
be small, even if you have collected a billion spam emails.

An alternative approach is to use Bayes' rule to get
\begin{equation}
P(S|\textbf{w})=\frac{P(\textbf{w}|S)P(S)}{P(\textbf{w})}
\end{equation}
This doesn't look any better, $P(\textbf{w}|S)$ is no easier to
estimate than $P(S|\textbf{w})$. However, in the na\"{i}ve Bayes
estimator it is additionally assumed that the different words are
independent so that 
\begin{eqnarray}
P((1,1,0,0,0,1)|S)&=&P(\mbox{enlargement}|S)P(\mbox{xxx}|S)[1-P(\mbox{cheapest}|S)]\times\cr&&[1-P(\mbox{pharmaceuticals}|S)][1-P(\mbox{satisfied}|S)]P(\mbox{leeds}|S)
\end{eqnarray}
This is clearly inaccurate, a spam email containing
\lq{}enlargement\rq{} is more likely to contain \lq{}satisfied\rq{}
than one that doesn't, that is why it is a \lq{}na\"ive\rq{}
classifier. The advantage though is that the individual probabilities
are much easier to estimate, there will be more emails with
\lq{}leeds\rq{} than there will be emails with the exact
combination of words represented by $(1,1,0,0,0,1)$ and so counting
occurrences will be much more accurate. The same approach can be used
to calculate $P(\textbf{w})$. Although the assumption that the words
are independent is not correct, these estimators are quite effective. 

\subsection*{Conditional probability}

Lets return to our slightly odd example of independence involving the
boring version of snakes and ladders. What is $P(S_1^1|S_2^1)$? In
other words, if you observe that someone is on the first square after
two moves, what is the probability that they were on the first square
after one move. Well
\begin{equation}
P(S_1^1|S_2^1)=\frac{P(S_2^1|S_1^1)P(S_1^1)}{P(S_2^1)}
\end{equation}
and we can calculate all these quantities: $P(S_2^1|S_1^1)=1/2$, and
$P(S_1^1)=1/2$ as well, as is $P(S_2^1)$, so
\begin{equation}
P(S_1^1|S_2^1)=1/2
\end{equation}
Similarly $P(S_3^2|S_2^1)=1/2$. Finally consider the probability
$P(S_3^2\cap S_1^1|S_2^1)$. This is the probability of the sequence
(H, T, H) when there are four sequence with $S_2^1$: (T, H, H), (T, H,
T), (H, T, H) and (H, T, T). Thus 
\begin{equation}
P(S_3^2\cap S_1^1|S_2^1)=1/4
\end{equation}
and hence
\begin{equation}
P(S_3^2\cap S_1^1|S_2^1)=P(S_1^1|S_2^1)P(S_3^2|S_2^1)
\end{equation}

This is an example of conditional independence: two events $A$ and $B$
are \textbf{conditionally independent} conditional on a third event
$C$ is
\begin{equation}
P(A\cap B|C)=P(A|C)P(B|C)
\end{equation}
In this case $A$ and $B$ might be related to each other, but only
through $C$, so if we know $C$ then $A$ and $B$ are independent. 

This applies to the snakes and ladders example; $S_1^1$ and $S_3^2$
are dependent. To work out the probability $P(S_1^1\cap S_3^2)$ we can
count the number of sequence of heads and tails this holds for:
(H,T,H) and (H,H,T), two out of eight possible sequences so
\begin{equation}
P(S_1^1\cap S_3^2)=1/4
\end{equation}
However, since $P(S_3^2)=3/8$ we see that 
\begin{equation}
P(S_1^1\cap S_3^2)\not = P(S_1^1)P(S_3^2)
\end{equation}
Thus $S_1^1$ and $S_3^2$ are not independence events. It is clear why
this is, the result after the first round affects the results after
two rounds, and that in turn affects the result after three. However
if we specify the result after two rounds, there is no further
dependence of the result after the first round and the result after
the third, they are related to each only through the result after the
second round.

The reason to mention this here is that it is part of the idea behind
a Markov chain. Markov chains go beyond this unit but roughly speaking
a Markov chain encodes a set of conditional independence relationships
between events. They are used to model many statistical processes. A
Markov chain model of language, for example, might assume the
probability of the next word in the sentence depends only on the
previous one. For example, if you look at the sentence \lq{}Don't
forget you are going to Aunt Alicia's\rq{} a Markov model asked to
predict the sixth word would look at the word \lq{}going\rq{} and use
the probability table for how often different words come after
\lq{}going\rq{}; it wouldn't consider the words coming earlier in the
sentence, the \lq{}Don't\rq{} and \lq{}forget\rq{} and so on. This
restriction is clearly useful from a computational point of view; by
assuming the word after \lq{}going\rq{} only depends on the earlier
words through the word \rq{}going\lq{} the problem of language
predictions is reduced to calculating a large, but not impossibly
large, probability table. This model doesn't work well, it is a very
primitive model of language. More sophisticated models, and until a
few years ago the best models of language, use a Hidden Markov Model,
which nonetheless is designed around assumptions about conditional
independence.

\ifind
\section*{Summary}
\else
\subsection*{9 Gauss distribution}
\fi

\begin{itemize}

\item The \textbf{Gau\ss{}ian distribution} has density
  \begin{equation}
p(x)=\frac{1}{\sqrt{2\pi\sigma^2}}e^{-\frac{(x-\mu)^2}{2\sigma^2}}
  \end{equation}


\item You can  shown that the mean is $\mu$ and the variance is $\sigma^2$ as the notation would suggest by differenciating
\begin{equation}
1=Z=\int_{-\infty}^\infty p(x)dx
\end{equation}
with respect to $\mu$.

%\item It has moment generating function
%  \begin{equation}
%m(t)= e^{\mu t + \frac{1}{2}\sigma^2 t^2}
%  \end{equation}
%  from which can be be shown that the mean is $\mu$ and the variance is $\sigma%^2$ as the notation would suggest.

\item To work out probabilities you need to use the \textbf{error function}
  \begin{equation}
\mbox{erf}\,(x)=\frac{1}{\sqrt{\pi}}\int_{-x}^xe^{-y^2}dy=\frac{2}{\sqrt{\pi}}\int_0^xe^{-y^2}dy
  \end{equation}
  In fact
  \begin{equation}
\mbox{Prob}(x_1<x<x_2)=\frac{1}{2}[\mbox{erf}\,(z_2)-\mbox{erf}\,(z_1)]
  \end{equation}
  where
  \begin{equation}
z=\frac{x-\mu}{\sqrt{2}\sigma}
  \end{equation}
  \end{itemize}


\ifind
\section*{Example question}
\else
\subsection*{2 conditional probability - example question}
\fi

This is known as the `second sibling' problem and like the Monty Hall problem it was popularized by Marilyn Vos Savant:
\begin{quote}
 A shopkeeper says she has two new baby beagles to show you, but
    she doesn't know whether they're male, female, or a pair. You tell
    her that you want only a male, and she telephones the fellow who's
    giving them a bath. "Is at least one a male?" she asks him. "Yes!"
    she informs you with a smile. What is the probability that the
    other one is a male?
\end{quote}

\noindent \textbf{solution} So we are assuming that males and females are equally likely and that the sex of the two pups is independent; the set of outcomes, using the obvious notation, is $\mathcal{X}=\{(m,m),(m,f),(f,m),(f,f)\}$ with each event having probability $1/4$. The event that at least one is male is $\mathcal{M}=\{(m,m),(m,f),(f,m)\}$. This means that $P(\mathcal{M})=3/4$. Lets call the event that the second dog is male $\mathcal{S}$ so $\mathcal{M}\cap\mathcal{S}=\{(m,m)\}$ and
\begin{equation}
P(\mathcal{M}\cap\mathcal{S})=\frac{1}{4}
\end{equation}
and hence
\begin{equation}
P(\mathcal{S}|\mathcal{M})=\frac{1}{3}
\end{equation}
and so the probability the second dog is male is one in three.




\end{document}


