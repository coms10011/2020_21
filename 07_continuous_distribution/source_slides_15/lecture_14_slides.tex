\documentclass{beamer}
\usepackage[latin1]{inputenc}
%\usetheme{Montpellier}
%\usetheme{Boadilla}
%\usecolortheme[RGB={204,51,255}]{structure}
%\usecolortheme[named=purple]{structure}
\usecolortheme[RGB={128,62,62}]{structure}
%\definecolor{dark}{rgb}{0.3,0.15,0.3}
%\definecolor{light}{rgb}{0.8,0.6,0.8}
%\definecolor{reddish}{rgb}{.5,0.15,0.15}
\definecolor{dark}{rgb}{0.5,0.3,0.4}
%\definecolor{light}{rgb}{0.8,0.6,0.8}
\definecolor{reddish}{rgb}{.7,0.25,0.25}
\definecolor{greenish}{rgb}{.25,0.7,0.25}
\definecolor{blueish}{rgb}{.25,0.25,0.7}
\definecolor{purple}{rgb}{.5,0.0,0.5}
\usepackage{graphicx}
\usepackage{pstricks}

\usepackage{amssymb}

\usepackage{amsmath}
\setbeamertemplate{navigation symbols}{}

\newcommand{\crish}{\color{reddish}}
\newcommand{\cbla}{\color{black}}
\newcommand{\cred}{\color{red}}
\newcommand{\cblu}{\color{blue}}
\newcommand{\cgre}{\color{green}}

\newcommand{\sm}{\color{reddish}$}
\newcommand{\fm}{$\color{black}}
\usepackage{tikz}
\usetikzlibrary{arrows,decorations.markings,positioning}
\usetikzlibrary{calc,fit,shapes, backgrounds} 

\usepackage{epstopdf}
\title{Lecture 15: Continuous distributions}
\author{COMS10014 Mathematics for Computer Science A}
\institute{\texttt{cs-uob.github.io/COMS10014/ and github.com/coms10011/2020\_21}}
\date{December 2020}
\begin{document}

\maketitle

\begin{frame}{Finding an average sized baby}
  \begin{center}
    \includegraphics[width=10cm]{baby_weigh.jpg}
  \end{center}
\end{frame}


\begin{frame}{Finding an average sized baby}
  \begin{center}
    \includegraphics[width=10cm]{baby_scale.jpg}
  \end{center}
\end{frame}

\begin{frame}{Probabilities}

  We don't mean `what is the probability of a baby weighing 2.75kg',
  we mean, what is the probability of a baby in a small range around 2.75kg:
  \crish$$P(W\in [2.745,2.755])$$\cbla{}

\end{frame}

\begin{frame}{Cumulative and density}

  We describe the probability distribution with a \textbf{distribution function} or \textbf{cumulative}:
\crish$$F(x)=P(X<x)$$\cbla{}
or a \textbf{density function}:
\crish$$f(x)=\frac{dF}{dx}$$\cbla{}
with
\crish$$
F(x)=\int_{-\infty}^x f(y)dy
$$\cbla{}
\end{frame}

\begin{frame}{Cumulative and density}
\crish$$F(x)=P(X<x)$$\cbla{}
with
\crish$$
F(x)=\int_{-\infty}^x f(y)dy
$$\cbla{}
so
\crish$$
\lim_{x\rightarrow \infty}F(x)=1
$$\cbla{}
or
\crish$$
\int_{-\infty}^\infty f(y)dy=1
$$\cbla{}
\end{frame}

\begin{frame}{Probabilities}
\crish$$
P(x\in [x_1,x_2])=P(x\le x_2)-P(x<x_1)
$$\cbla{}
For a continuous variable we don't have to be careful about the
distinction between \crish$x<x_2$\cbla{}  and \crish$x\le x_2$\cbla{}. Hence
\crish$$
P(x\in [x_1,x_2])=F(x_2)-F(x_1)
$$\cbla{}
or
\crish$$
P(x\in [x_1,x_2])=\int_{x_1}^{x_2} f(y)dy
$$\cbla{}
\end{frame}

\begin{frame}{Constant density}
\crish$$
f(x)=\left\{\begin{array}{ll}\frac{1}{2}&x\in [-1,1]\\0&\mbox{otherwise}\end{array}\right.
$$\cbla{}
\begin{center}
\include{fig_p_const}
\end{center}

\end{frame}


\begin{frame}{Constant density}
\crish$$
F(x)=\left\{\begin{array}{ll}0&x<-1\\
\frac{x+1}{2}&x\in [-1,1]\\1&x>1\end{array}\right.
$$\cbla{}

\begin{center}
\include{fig_c_const}
\end{center}

\end{frame}

\begin{frame}{Properties}
  \crish$F(x)\ge F(y)$\cbla{} if \crish$x>y$\cbla{} so it is monotonically increasing function so
  \crish$$p(x)=\frac{dF(x)}{dx}\ge 0$$\cbla{}
\end{frame}

\begin{frame}{Properties}
  Clearly
\crish$$
  P(x\in [a,b])=\int_a^bf(y)dy\le 1
$$\cbla{}
  but \crish$p(x)$\cbla{} can be greater than one. For example
\crish$$
f(x)=\left\{\begin{array}{ll}4&x\in [0,1/4]\\0&\mbox{otherwise}\end{array}\right.
$$\cbla{}
\end{frame}
  

\begin{frame}{Properties}
\crish$$
f(x)=\left\{\begin{array}{ll}4&x\in [0,1/4]\\0&\mbox{otherwise}\end{array}\right.
$$\cbla{}

\begin{center}
\include{narrow}
\end{center}

\end{frame}




\end{document}

